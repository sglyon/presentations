% -*- program: xelatex -*-
% \documentclass[10pt, compress, handout]{beamer}
\documentclass[compress,10pt]{beamer}
\usetheme[usetitleprogressbar]{m}


\usepackage{longtable,booktabs}
\usepackage[scale=2]{ccicons}
\usepackage{minted}
\usepackage{caption}
\usepackage{hyperref} % Used to add links to websites to the pdf
\definecolor{mLightBrown}{HTML}{EB811B}
\definecolor{TolColor1}{HTML}{332288}   % dark purple
\definecolor{TolColor2}{HTML}{6699CC}   % dark blue
\definecolor{TolColor4}{HTML}{44AA99}   % light green
\definecolor{TolColor8}{HTML}{661100}   % dark red
\hypersetup{colorlinks,linkcolor=TolColor8,urlcolor=mLightBrown,citecolor=black}  % From Rick
% These lines are needed to make table captions work with longtable:
\makeatletter
\def\fnum@table{\tablename~\thetable}
\makeatother



\usepgfplotslibrary{dateplot}
\usemintedstyle{trac}

\newcommand{\TODO}[1]{ {\color{magenta} TODO: #1} }
% fraction with parenthesis around it
\newcommand{\pfrac}[2]{\ensuremath{ \left( \frac{#1}{#2} \right)}}
\newcommand{\var}[0]{ \text{Var} }
\newcommand{\cov}[0]{ \text{Cov} }


\title{Julian Economics}
\author{Spencer Lyon}

\date{\today}

\begin{document}

\begin{frame}[fragile]
  \titlepage
\end{frame}

\begin{frame} \frametitle{Outline}
\tableofcontents[hideallsubsections]
\end{frame}

\section{Intro}\label{intro}

\begin{frame}{About me}

\begin{itemize}
\itemsep1pt\parskip0pt\parsep0pt
\item
  Personal

  \begin{itemize}
  \itemsep1pt\parskip0pt\parsep0pt
  \item
    Economics PhD student at NYU.
  \item
    Physics and econ undergrad
  \item
    I have a wife and two kids
  \end{itemize}
\item
  Programming

  \begin{itemize}
  \itemsep1pt\parskip0pt\parsep0pt
  \item
    Started on Mathematica
  \item
    First love was Python
  \item
    Dabbled with C, C++, R, Scala, Haskell, MATLAB
  \item
    New favorite for many tasks is Julia
  \end{itemize}
\end{itemize}

\end{frame}

\begin{frame}{Why I like Julia}

\begin{itemize}
\itemsep1pt\parskip0pt\parsep0pt
\item
  Fast
\item
  Functional
\item
  Flexible
\item
  Clean
\item
  Open source
\end{itemize}

\end{frame}

\begin{frame}{Julia is \ldots{} fast}

\begin{itemize}
\itemsep1pt\parskip0pt\parsep0pt
\item
  We've seen the \href{http://julialang.org}{benchmarks}
\item
  This matters for economics because problems often have:

  \begin{itemize}
  \itemsep1pt\parskip0pt\parsep0pt
  \item
    Many states
  \item
    Solving functional equations on state space
  \item
    Many algorithms require explicit looping over matrices that
    represent these functions
  \item
    Typical model is stochastic \(\Rightarrow\) requires approximation
    of expectation \(\Rightarrow\) hard (impossible?) to parallelize
    \emph{across} iterations
  \end{itemize}
\item
  So, fast iterations are crucial
\end{itemize}

\end{frame}

\begin{frame}{Julia is \ldots{} functional}

\begin{itemize}
\itemsep1pt\parskip0pt\parsep0pt
\item
  Proper support of basic functional programming makes code readable and
  concise:

  \begin{itemize}
  \itemsep1pt\parskip0pt\parsep0pt
  \item
    \texttt{do} notation
  \item
    \texttt{map}, \texttt{fold(l\textbar{}r)}, \texttt{reduce},
    \texttt{pmap}, comprehensions, \ldots{}
  \end{itemize}
\item
  ``Litetweight'' types make it natural to have very small types (can be
  treated like a Dict in python or a list in R, with the additional
  ability to specify how functions operate on it, even relative to types
  of neighboring arguments)
\item
  Multiple dispatch lets you combine two previous points in unique and
  powerful ways (e.g.~type-based API -- not kwarg. Example to come)
\end{itemize}

\end{frame}

\begin{frame}{Julia is \ldots{} flexible}

\begin{itemize}[<+->]
\itemsep1pt\parskip0pt\parsep0pt
\item
  Most of the standard library written in Julia itself -- and it is fast

  \begin{itemize}[<+->]
  \itemsep1pt\parskip0pt\parsep0pt
  \item
    Means other code written in Julia has \emph{potential} to perform at
    the same level as the standard library (if written well)
  \item
    Not true of most high level languages (e.g.~Python, R, Matlab) where
    standard library or core numeric tools written in lower level
    language
  \item
    To achieve standard library performance in those languages you
    usually need to write routines in C/C++/Fortran and wrap it for use
    in high-level language
  \end{itemize}
\item
  Can call C, natively with no overhead and no wrappers.

  \begin{itemize}[<+->]
  \itemsep1pt\parskip0pt\parsep0pt
  \item
    Opens door to call
    \href{http://localhost:8888/notebooks/PyCall.ipynb}{Python}
    (PyCall.jl), \href{http://localhost:8888/notebooks/rcall.ipynb}{R}
    (RCall.jl), MATLAB (MATLAB.jl), Java (JavaCall.jl), ect.
  \item
    Easily write Julia interface into mature C libraries (LAPACK and
    BLAS in standard library, NLopt, Sundials, many more\ldots{})
  \end{itemize}
\end{itemize}

\end{frame}

\begin{frame}{Julia is \ldots{} clean and open source}

\begin{itemize}
\itemsep1pt\parskip0pt\parsep0pt
\item
  Syntax is powerful and concise

  \begin{itemize}
  \itemsep1pt\parskip0pt\parsep0pt
  \item
    Convenient linear algebra syntax (\texttt{A\ *\ B} instead of
    \texttt{A\ \%*\%\ B} or \texttt{np.dot(A,\ B)})
  \item
    Matlab-esque matrix construction
  \item
    Minor points, but make the experience better
  \end{itemize}
\item
  Open source

  \begin{itemize}
  \itemsep1pt\parskip0pt\parsep0pt
  \item
    Learn how (and sometimes why) standard library functions are
    implemented
  \item
    Gihub issue list or the google group great ways to watch progress
  \end{itemize}
\end{itemize}

\end{frame}

\section{Quant-Econ}\label{quant-econ}

\begin{frame}{Website}

\begin{quote}
\href{http://quantecon.org}{QuantEcon} is an organization run by
economists for economists with the aim of coordinating distributed
development of high quality open source code for all forms of
quantitative economic modeling.
\end{quote}

\begin{itemize}
\itemsep1pt\parskip0pt\parsep0pt
\item
  Two fold:

  \begin{enumerate}
  \def\labelenumi{\arabic{enumi}.}
  \itemsep1pt\parskip0pt\parsep0pt
  \item
    \href{http://quant-econ.net}{Website} with over almost 40 teaching
    modules (textbook chapters) that teach programming and economics
  \item
    Code libraries in
    \href{https://github.com/QuantEcon/QuantEcon.py}{Python} and
    \href{https://github.com/QuantEcon/QuantEcon.jl}{Julia}
  \end{enumerate}
\end{itemize}

\end{frame}

\begin{frame}{Libraries}

\begin{itemize}
\itemsep1pt\parskip0pt\parsep0pt
\item
  Started as teaching tools -- implementations of routines in chapters
\item
  Transitioning into performance-oriented set of tools
\item
  Julia and Python versions, both first class members
\item
  Open source, community developed, on github
\end{itemize}

\end{frame}

\section{Examples}\label{examples}

\begin{frame}{Playing nicely}

\begin{itemize}
\itemsep1pt\parskip0pt\parsep0pt
\item
  Many potential users may be worried about ``abandoning'' code they
  have written or rely on from other languages
\item
  Julia's ability to naturally call R and Python might alleviate these
  concerns
\item
  NOTE: show RCall and PyCall examples
\end{itemize}

\end{frame}

\begin{frame}{Functional Style}

\begin{itemize}
\itemsep1pt\parskip0pt\parsep0pt
\item
  As mentioned before; Julia's functional style, lightweight types, and
  multiple dispatch open the door for unique API design opportunities
\item
  NOTE: show
  \href{https://github.com/spencerlyon2/IterationManagers.jl}{IterationManagers}
  and \href{https://github.com/spencerlyon2/CompEcon.jl}{CompEcon}
  examples
\end{itemize}

\end{frame}

\section{Pitfalls}\label{pitfalls}

\begin{frame}{Spaghetti}

\begin{itemize}
\itemsep1pt\parskip0pt\parsep0pt
\item
  Lightweight types and multiple dispatch offer power
\item
  \ldots{}sometimes too much power
\item
  It is tempting to make a type or new function for every operation
  ``just in case'' you will dispatch on it in the future
\item
  You might end up with complicated spaghetti code composed of many
  \texttt{type}s of noodles
\item
  Personal rule(s) of thumb

  \begin{itemize}
  \itemsep1pt\parskip0pt\parsep0pt
  \item
    \textbf{Don't} break out parts of a function if I will only ever
    call them from one place
  \item
    \textbf{Do} break out parts of a function if I can dispatch on them
  \end{itemize}
\end{itemize}

\end{frame}

\begin{frame}{Young language}

\begin{itemize}
\itemsep1pt\parskip0pt\parsep0pt
\item
  Community is relatively small (compared to Python, R, MATLAB)

  \begin{itemize}
  \itemsep1pt\parskip0pt\parsep0pt
  \item
    Cons

    \begin{itemize}
    \itemsep1pt\parskip0pt\parsep0pt
    \item
      Not many mature learning materials
    \item
      Less collective man power writing packages, tutorials, ect.
    \end{itemize}
  \item
    Pros

    \begin{itemize}
    \itemsep1pt\parskip0pt\parsep0pt
    \item
      Interact with ``big'' names (``Hi Stefan!'' :) )
    \item
      Opportunity for users to help form community, culture, even the
      language itself
    \end{itemize}
  \end{itemize}
\item
  Not to version 1.0 yet

  \begin{itemize}
  \itemsep1pt\parskip0pt\parsep0pt
  \item
    (quickly) Moving target to develop against
  \item
    Code that ran a few months ago might not run today
  \item
    Usually very easy to fix these issues
  \end{itemize}
\item
  Conventions still in flux (docs, testing, style-guide not quite PEP8)
\item
  Package ecosystem not as rich as Python or R (or MATLABs toolboxes for
  specific functionality)

  \begin{itemize}
  \itemsep1pt\parskip0pt\parsep0pt
  \item
    But it is growing\ldots{}
    \href{http://pkg.julialang.org/pulse.html}{fast}
  \end{itemize}
\end{itemize}

\end{frame}

\begin{frame}{Convex learning curve}

\begin{itemize}
\itemsep1pt\parskip0pt\parsep0pt
\item
  Easy to learn, hard to master
\item
  MATLAB or NumPy users immediately comfortable writing functions and
  using Arrays
\item
  Unlocking full Julia potential requires

  \begin{itemize}
  \itemsep1pt\parskip0pt\parsep0pt
  \item
    Learning to think functionally (not traditional OOP, or even
    procedural)
  \item
    Understanding type system (abstract, composite, parametric\ldots{})
    can be intimidating
  \item
    Advanced features like meta-programming are powerful and seductive,
    but often improperly used
  \end{itemize}
\end{itemize}

\end{frame}

\section{Final thoughts}\label{final-thoughts}

\begin{frame}{Tips}

\begin{itemize}
\itemsep1pt\parskip0pt\parsep0pt
\item
  \texttt{@less}, \texttt{@which}, \texttt{JULIA\_EDITOR} +
  \texttt{@edit}
\item
  Get involved - follow mailing list or issue list on github
\end{itemize}

\end{frame}

\end{document}
